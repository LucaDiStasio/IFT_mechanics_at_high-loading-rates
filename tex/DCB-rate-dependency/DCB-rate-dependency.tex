\documentclass[review]{elsarticle}

\usepackage{amsmath}
\usepackage{booktabs}
\usepackage{tabularx}
\usepackage{lineno,hyperref}
\modulolinenumbers[5]

\journal{Project proposal}

%%%%%%%%%%%%%%%%%%%%%%%
%% Elsevier bibliography styles
%%%%%%%%%%%%%%%%%%%%%%%
%% To change the style, put a % in front of the second line of the current style and
%% remove the % from the second line of the style you would like to use.
%%%%%%%%%%%%%%%%%%%%%%%

%% Numbered
%\bibliographystyle{model1-num-names}

%% Numbered without titles
%\bibliographystyle{model1a-num-names}

%% Harvard
%\bibliographystyle{model2-names.bst}\biboptions{authoryear}

%% Vancouver numbered
%\usepackage{numcompress}\bibliographystyle{model3-num-names}

%% Vancouver name/year
%\usepackage{numcompress}\bibliographystyle{model4-names}\biboptions{authoryear}

%% APA style
%\bibliographystyle{model5-names}\biboptions{authoryear}

%% AMA style
%\usepackage{numcompress}\bibliographystyle{model6-num-names}

%% `Elsevier LaTeX' style
\bibliographystyle{elsarticle-num}
%%%%%%%%%%%%%%%%%%%%%%%

\begin{document}

\begin{frontmatter}

\title{Experimental and numerical assessment of loading rate effects on Mode I delamination in carbon fiber/epoxy composites}
%\tnotetext[mytitlenote]{Fully documented templates are available in the elsarticle package on \href{http://www.ctan.org/tex-archive/macros/latex/contrib/elsarticle}{CTAN}.}

%% Group authors per affiliation:
%\author{Luca Di Stasio\fnref{myfootnote}}
%\address{Radarweg 29, Amsterdam}
%\fntext[myfootnote]{Since 1880.}

%% or include affiliations in footnotes:
\author[nancy,lulea]{Luca Di Stasio}
\author[lulea]{Janis Varna}
\author[nancy]{Zoubir Ayadi}
%\ead[url]{www.elsevier.com}

%\author[mysecondaryaddress]{Global Customer Service\corref{mycorrespondingauthor}}
%\cortext[mycorrespondingauthor]{Corresponding author}
%\ead{support@elsevier.com}

\address[nancy]{Universit\'e de Lorraine, EEIGM, IJL, 6 Rue Bastien Lepage, F-54010 Nancy, France}
\address[lulea]{Lule\aa\ University of Technology, University Campus, SE-97187 Lule\aa, Sweden}
\address[madrid]{IMDEA Materials Institute, Getafe, Madrid, Spain}
\begin{abstract}
\noindent

\end{abstract}

%\begin{keyword}
%\texttt{elsarticle.cls}\sep \LaTeX\sep Elsevier \sep template
%\MSC[2010] 00-01\sep  99-00
%\end{keyword}

\end{frontmatter}

\linenumbers

\section{Introduction}



\section{Objectives}

\begin{table}[!h]
\centering
\begin{tabularx}{\textwidth}{ccccX}
$v\ \left[\frac{mm}{min}\right]$&$\varepsilon\ \left[\%\right]$&\multicolumn{2}{c}{$T\ \left[^{\circ}\right]$}&Estimated time (no counting) for $L=50 \left[mm\right]$, $\left[min\right]$\\
&&$\sim 20$ (room)&120&\\
\toprule
\midrule
1&0.2&$E_{L},\rho_{c},d$&$E_{L},\rho_{c},d$&0.5\\
&0.4&$E_{L},\rho_{c},d$&$E_{L},\rho_{c},d$&0.6\\
&0.8&$E_{L},\rho_{c},d$&$E_{L},\rho_{c},d$&0.8\\
&1.0&$E_{L},\rho_{c},d$&$E_{L},\rho_{c},d$&0.9\\
&1.2&$E_{L},\rho_{c},d$&$E_{L},\rho_{c},d$&1\\
\midrule
10&0.2&$E_{L},\rho_{c},d$&$E_{L},\rho_{c},d$&\\
&0.4&$E_{L},\rho_{c},d$&$E_{L},\rho_{c},d$&\\
&0.8&$E_{L},\rho_{c},d$&$E_{L},\rho_{c},d$&\\
&1.0&$E_{L},\rho_{c},d$&$E_{L},\rho_{c},d$&\\
&1.2&$E_{L},\rho_{c},d$&$E_{L},\rho_{c},d$&\\
\midrule
50&0.2&$E_{L},\rho_{c},d$&$E_{L},\rho_{c},d$&\\
&0.4&$E_{L},\rho_{c},d$&$E_{L},\rho_{c},d$&\\
&0.8&$E_{L},\rho_{c},d$&$E_{L},\rho_{c},d$&\\
&1.0&$E_{L},\rho_{c},d$&$E_{L},\rho_{c},d$&\\
&1.2&$E_{L},\rho_{c},d$&$E_{L},\rho_{c},d$&\\
\midrule
500&0.2&$E_{L},\rho_{c},d$&$E_{L},\rho_{c},d$&\\
&0.4&$E_{L},\rho_{c},d$&$E_{L},\rho_{c},d$&\\
&0.8&$E_{L},\rho_{c},d$&$E_{L},\rho_{c},d$&\\
&1.0&$E_{L},\rho_{c},d$&$E_{L},\rho_{c},d$&\\
&1.2&$E_{L},\rho_{c},d$&$E_{L},\rho_{c},d$&\\
\end{tabularx}
\end{table}

\section{Materials}

If all strain levels are applied to the same specimens: $8n$ specimens need to be tested, where $n$ is the number of measurements for the same combination of parameters.\\
If each strain level is applied to one specimen: $40n$ specimens need to be tested, where $n$ is the number of measurements for the same combination of parameters.\\ 
If each strain level is applied to one specimen only at high $T$: $28n$ specimens need to be tested, where $n$ is the number of measurements for the same combination of parameters.\\ 

\section{Methods}



\section{Expected outcomes}

\section{Audience}

Students attending the Aerospace Materials course.

\end{document}
